% vim: set tw=78 sts=2 sw=2 ts=8 aw et ai:

The concept of virtualization first emerged in the 1960, when IBM developed the first operating systems which could run another operating system as a process\cite{IBM}. Since then, the subject has seen considerable progress. Most importantly, it has driven the development of future hardware platforms, with features dedicated to virtualization being present in many modern processors.

The evolution of embedded hardware have made this type of hardware suitable for many functions which were only performed in the past by desktop or server systems. These functions vary greatly, but are generally centred around server-like behaviour.

However, the advances made in hardware are not enough to qualify embedded devices as a viable alternative for traditional server hardware. Virtualization is necessary to enable separation between multiple execution environments. The main reasons for this are distributing functionality across various components, providing security through isolation and having the possibility of running software released under different licenses on the same machine. Moreover, it provides a level of hardware abstraction, resulting in less effort to make software portable between platforms.

ARM processors are a prime example of the evolution of embedded systems. The v7 family of CPUs possesses specialized extensions that facilitate virtualization.
